    \documentclass{article}
    \usepackage{amsmath}
    \usepackage{booktabs}
    \usepackage[margin=1in]{geometry}
    \usepackage{titling}
    \usepackage{tikz}
    \usepackage{graphicx}
    \usepackage[margin=1in]{geometry}
    \usetikzlibrary{matrix}
    \usepackage{pdfpages}
    
    \setlength{\droptitle}{-4em}
    
    \title{\textbf{Lab Report 8 – Design and Implementation of a Digital Up/Down Counter from 0 to 99}}
    \author{
    Arnav Yadnopavit -- EE24BTECH11007\\
    Prajwal -- EE24BTECH11051
    }
    \date{}
    
    \begin{document}
    
    \maketitle
    \section{First digit}
    \section*{BCD Up Counter State Transition Table (0 to 9 with Don't Cares)}
    
    \begin{center}
    \begin{tabular}{cccc|cccc|cccc|c}
    \toprule
    \multicolumn{4}{c|}{Present State} & \multicolumn{4}{c|}{Next State} & \multicolumn{4}{c|}{T Inputs} & Remark \\
    $Q_3$ & $Q_2$ & $Q_1$ & $Q_0$ & $Q_3^+$ & $Q_2^+$ & $Q_1^+$ & $Q_0^+$ & $T_3$ & $T_2$ & $T_1$ & $T_0$ & \\
    \midrule
    0 & 0 & 0 & 0 & 0 & 0 & 0 & 1 & 0 & 0 & 0 & 1 & 0 $\rightarrow$ 1 \\
    0 & 0 & 0 & 1 & 0 & 0 & 1 & 0 & 0 & 0 & 1 & 1 & 1 $\rightarrow$ 2 \\
    0 & 0 & 1 & 0 & 0 & 0 & 1 & 1 & 0 & 0 & 0 & 1 & 2 $\rightarrow$ 3 \\
    0 & 0 & 1 & 1 & 0 & 1 & 0 & 0 & 0 & 1 & 1 & 1 & 3 $\rightarrow$ 4 \\
    0 & 1 & 0 & 0 & 0 & 1 & 0 & 1 & 0 & 0 & 0 & 1 & 4 $\rightarrow$ 5 \\
    0 & 1 & 0 & 1 & 0 & 1 & 1 & 0 & 0 & 0 & 1 & 1 & 5 $\rightarrow$ 6 \\
    0 & 1 & 1 & 0 & 0 & 1 & 1 & 1 & 0 & 0 & 0 & 1 & 6 $\rightarrow$ 7 \\
    0 & 1 & 1 & 1 & 1 & 0 & 0 & 0 & 1 & 1 & 1 & 1 & 7 $\rightarrow$ 8 \\
    1 & 0 & 0 & 0 & 1 & 0 & 0 & 1 & 0 & 0 & 0 & 1 & 8 $\rightarrow$ 9 \\
    1 & 0 & 0 & 1 & 0 & 0 & 0 & 0 & 1 & 0 & 0 & 1 & 9 $\rightarrow$ 0 \\
    \midrule
    1 & 0 & 1 & 0 & X & X & X & X & X & X & X & X & Don't Care \\
    1 & 0 & 1 & 1 & X & X & X & X & X & X & X & X & Don't Care \\
    1 & 1 & 0 & 0 & X & X & X & X & X & X & X & X & Don't Care \\
    1 & 1 & 0 & 1 & X & X & X & X & X & X & X & X & Don't Care \\
    1 & 1 & 1 & 0 & X & X & X & X & X & X & X & X & Don't Care \\
    1 & 1 & 1 & 1 & X & X & X & X & X & X & X & X & Don't Care \\
    \bottomrule
    \end{tabular}
    \end{center}
    \section*{Karnaugh Maps for T Flip-Flop Inputs}
    
    \subsection*{Up Counter (Controlled by $B_u \cdot \overline{B_d}$)}
    
    \textbf{$T_0$ (Up)}  
    \[
    (T_0)_{up} = B_u \cdot \overline{B_d} \cdot 1 = B_u \cdot \overline{B_d}
    \]
    
    \begin{center}
    \begin{tikzpicture}
    \matrix[matrix of nodes,nodes={draw, minimum size=1cm, anchor=center},column sep=-\pgflinewidth, row sep=-\pgflinewidth] (m) {
        ~ & 00 & 01 & 11 & 10 \\
      00 & 1 & 1 & 1 & 1 \\
      01 & 1 & 1 & 1 & 1 \\
      11 & 1 & 1 & 1 & 1 \\
      10 & 1 & 1 & 1 & 1 \\
    };
    \draw (m-1-1.south west) rectangle (m-5-5.north east);
    \end{tikzpicture}
    \end{center}
    \vspace{1cm}
    
    \textbf{$T_1$ (Up)}  
    \[
    (T_1)_{up} = Q_0 \cdot \overline{Q_3} \cdot B_u \cdot \overline{B_d}
    \]
    
    \begin{center}
    \begin{tikzpicture}
    \matrix[matrix of nodes,nodes={draw, minimum size=1cm, anchor=center},column sep=-\pgflinewidth, row sep=-\pgflinewidth] (m) {
        ~ & 00 & 01 & 11 & 10 \\
      00 & 0 & 1 & 1 & 0 \\
      01 & 0 & 1 & 1 & 0 \\
      11 & X & X & X & X \\
      10 & 0 & 0 & X & X \\
    };
    \draw (m-1-1.south west) rectangle (m-5-5.north east);
    \end{tikzpicture}
    \end{center}
    \vspace{1cm}
    
    
    
    
    \textbf{$T_2$ (Up)}  
    \[
    (T_2)_{up} = Q_0 \cdot Q_1 \cdot B_u \cdot \overline{B_d}
    \]
    
    \begin{center}
    \begin{tikzpicture}
    \matrix[matrix of nodes,nodes={draw, minimum size=1cm, anchor=center},column sep=-\pgflinewidth, row sep=-\pgflinewidth] (m) {
        ~ & 00 & 01 & 11 & 10 \\
      00 & 0 & 0 & 1 & 0 \\
      01 & 0 & 0 & 1 & 0 \\
      11 & X & X & X & X \\
      10 & 0 & 0 & X & X \\
    };
    \draw (m-1-1.south west) rectangle (m-5-5.north east);
    \end{tikzpicture}
    \end{center}
    \vspace{1cm}
    
    
    
    
    \textbf{$T_3$ (Up)}  
    \[
    (T_3)_{up} = Q_0 \cdot (Q_3 +Q_2Q_1) \cdot B_u \cdot \overline{B_d}
    \]
    
    \begin{center}
    \begin{tikzpicture}
    \matrix[matrix of nodes,nodes={draw, minimum size=1cm, anchor=center},column sep=-\pgflinewidth, row sep=-\pgflinewidth] (m) {
        ~ & 00 & 01 & 11 & 10 \\
      00 & 0 & 0 & 0 & 0 \\
      01 & 0 & 0 & 1 & 0 \\
      11 & X & X & X & X \\
      10 & 0 & 1 & X & X \\
    };
    \draw (m-1-1.south west) rectangle (m-5-5.north east);
    \end{tikzpicture}
    \end{center}
    \vspace{1cm}
    
    
    
    
    
    
    \section*{BCD Down Counter State Transition Table (9 to 0 with Don't Cares)}
    
    \begin{center}
    \begin{tabular}{cccc|cccc|cccc|c}
    \toprule
    \multicolumn{4}{c|}{Present State} & \multicolumn{4}{c|}{Next State} & \multicolumn{4}{c|}{T Inputs} & Remark \\
    $Q_3$ & $Q_2$ & $Q_1$ & $Q_0$ & $Q_3^-$ & $Q_2^-$ & $Q_1^-$ & $Q_0^-$ & $T_3$ & $T_2$ & $T_1$ & $T_0$ & \\
    \midrule
    1 & 0 & 0 & 1 & 1 & 0 & 0 & 0 & 0 & 0 & 0 & 1 & 9 $\rightarrow$ 8 \\
    1 & 0 & 0 & 0 & 0 & 1 & 1 & 1 & 1 & 1 & 1 & 1 & 8 $\rightarrow$ 7 \\
    0 & 1 & 1 & 1 & 0 & 1 & 1 & 0 & 0 & 0 & 0 & 1 & 7 $\rightarrow$ 6 \\
    0 & 1 & 1 & 0 & 0 & 1 & 0 & 1 & 0 & 0 & 1 & 1 & 6 $\rightarrow$ 5 \\
    0 & 1 & 0 & 1 & 0 & 1 & 0 & 0 & 0 & 0 & 0 & 1 & 5 $\rightarrow$ 4 \\
    0 & 1 & 0 & 0 & 0 & 0 & 1 & 1 & 0 & 1 & 1 & 1 & 4 $\rightarrow$ 3 \\
    0 & 0 & 1 & 1 & 0 & 0 & 1 & 0 & 0 & 0 & 0 & 1 & 3 $\rightarrow$ 2 \\
    0 & 0 & 1 & 0 & 0 & 0 & 0 & 1 & 0 & 0 & 1 & 1 & 2 $\rightarrow$ 1 \\
    0 & 0 & 0 & 1 & 0 & 0 & 0 & 0 & 0 & 0 & 0 & 1 & 1 $\rightarrow$ 0 \\
    0 & 0 & 0 & 0 & 1 & 0 & 0 & 1 & 1 & 0 & 0 & 1 & 0 $\rightarrow$ 9 \\
    \midrule
    1 & 0 & 1 & 0 & X & X & X & X & X & X & X & X & Don't Care \\
    1 & 0 & 1 & 1 & X & X & X & X & X & X & X & X & Don't Care \\
    1 & 1 & 0 & 0 & X & X & X & X & X & X & X & X & Don't Care \\
    1 & 1 & 0 & 1 & X & X & X & X & X & X & X & X & Don't Care \\
    1 & 1 & 1 & 0 & X & X & X & X & X & X & X & X & Don't Care \\
    1 & 1 & 1 & 1 & X & X & X & X & X & X & X & X & Don't Care \\
    \bottomrule
    \end{tabular}
    \end{center}
    
    
    \subsection*{Down Counter (Controlled by $\overline{B_u} \cdot B_d$)}
    
    \textbf{$T_0$ (Down)}  
    \[
    T_0 = \overline{B_u} \cdot B_d
    \]
    
    \begin{center}
    \begin{tikzpicture}
    \matrix[matrix of nodes,nodes={draw, minimum size=1cm, anchor=center},column sep=-\pgflinewidth, row sep=-\pgflinewidth] (m) {
        ~ & 00 & 01 & 11 & 10 \\
      00 & 1 & 1 & 1 & 1 \\
      01 & 1 & 1 & 1 & 1 \\
      11 & 1 & 1 & 1 & 1 \\
      10 & 1 & 1 & 1 & 1 \\
    };
    \draw (m-1-1.south west) rectangle (m-5-5.north east);
    \end{tikzpicture}
    \end{center}
    \vspace{1cm}
    
    
    \textbf{$T_1$ (Down)}  
    \[
    T_1 = \overline{Q_0} \cdot (Q_1 + Q_2+Q_3) \cdot \overline{B_u} \cdot B_d
    \]
    
    \begin{center}
    \begin{tikzpicture}
    \matrix[matrix of nodes,nodes={draw, minimum size=1cm, anchor=center},column sep=-\pgflinewidth, row sep=-\pgflinewidth] (m) {
        ~ & 00 & 01 & 11 & 10 \\
      00 & 0 & 0 & 0 & 1 \\
      01 & 1 & 0 & 0 & 1 \\
      11 & X & X & X & X \\
      10 & 1 & 0 & X & X \\
    };
    \draw (m-1-1.south west) rectangle (m-5-5.north east);
    \end{tikzpicture}
    \end{center}
    \vspace{1cm}
    
    
    \textbf{$T_2$ (Down)}  
    \[
    T_2 = \overline{Q_1} \cdot \overline{Q_0} \cdot (Q_2 + Q_3) \cdot \overline{B_u} \cdot B_d
    \]
    
    \begin{center}
    \begin{tikzpicture}
    \matrix[matrix of nodes,nodes={draw, minimum size=1cm, anchor=center},column sep=-\pgflinewidth, row sep=-\pgflinewidth] (m) {
        ~ & 00 & 01 & 11 & 10 \\
      00 & 0 & 0 & 0 & 0 \\
      01 & 1 & 0 & 0 & 0 \\
      11 & X & X & X & X \\
      10 & 1 & 0 & X & X \\
    };
    \draw (m-1-1.south west) rectangle (m-5-5.north east);
    \end{tikzpicture}
    \end{center}
    \vspace{1cm}
    
    
    \textbf{$T_3$ (Down)}  
    \[
    T_3 = \overline{Q_2} \cdot \overline{Q_1} \cdot \overline{Q_0} \cdot \overline{B_u} \cdot B_d
    \]
    
    \begin{center}
    \begin{tikzpicture}
    \matrix[matrix of nodes,nodes={draw, minimum size=1cm, anchor=center},column sep=-\pgflinewidth, row sep=-\pgflinewidth] (m) {
        ~ & 00 & 01 & 11 & 10 \\
      00 & 1 & 0 & 0 & 0 \\
      01 & 0 & 0 & 0 & 0 \\
      11 & X & X & X & X \\
      10 & 1 & 0 & X & X \\
    };
    \draw (m-1-1.south west) rectangle (m-5-5.north east);
    \end{tikzpicture}
    \end{center}
    \vspace{1cm}
    \newpage
    \textbf{Combined T Flip-Flop Expressions (Up and Down Modes)}
    \vspace{0.5cm}
    
    \textbf{$T_0$ (Overall)}  
    \[
    T_0 = B_u \cdot \overline{B_d} + \overline{B_u} \cdot B_d
    \]
    \vspace{0.5cm}
    
    \textbf{$T_1$ (Overall)}  
    \[
    T_1 = Q_0 \cdot \overline{Q_3} \cdot B_u \cdot \overline{B_d} + \overline{Q_0} \cdot (Q_1 + Q_2 + Q_3) \cdot \overline{B_u} \cdot B_d
    \]
    \vspace{0.5cm}
    
    \textbf{$T_2$ (Overall)}  
    \[
    T_2 = Q_0 \cdot Q_1 \cdot B_u \cdot \overline{B_d} + \overline{Q_1} \cdot \overline{Q_0} \cdot (Q_2 + Q_3) \cdot \overline{B_u} \cdot B_d
    \]
    \vspace{0.5cm}
    
    \textbf{$T_3$ (Overall)}  
    \[
    T_3 = Q_0 \cdot (Q_3 + Q_2 Q_1) \cdot B_u \cdot \overline{B_d} + \overline{Q_2} \cdot \overline{Q_1} \cdot \overline{Q_0} \cdot \overline{B_u} \cdot B_d
    \]
    
    \section{\textbf{Second Digit}}
    We know that the state transition table and k-maps for the second digit will be same but we have to ensure that second digit only changes when first digit either transits form 9 to 0 or 0 to 9
    i,e
    \begin{center}
     \begin{tabular}{cccc|cccc|cccc|c}
    \toprule
    \multicolumn{4}{c|}{Present State} & \multicolumn{4}{c|}{Next State} & \multicolumn{4}{c|}{T Inputs} & Remark \\
    $Q_3$ & $Q_2$ & $Q_1$ & $Q_0$ & $Q_3$ & $Q_2$ & $Q_1$ & $Q_0$ & $T_3$ & $T_2$ & $T_1$ & $T_0$ & \\
    \midrule
    0 & 0 & 0 & 0 & 1 & 0 & 0 & 1 & 1 & 0 & 0 & 1 & 0 $\rightarrow$ 9 \\
    1 & 0 & 0 & 1 & 0 & 0 & 0 & 0 & 1 & 0 & 0 & 1 & 9 $\rightarrow$ 0 \\
    
    \bottomrule
    \end{tabular}   
    \end{center}
    \textbf{Therefore},\\
    \begin{center}
        $T'_0=(T_0)_{up} \cdot \overline{Q_3} \cdot \overline{Q_0} + (T_0)_{down} \cdot \overline{Q_3} \cdot \overline{Q_2} \cdot \overline{Q_1} \cdot \overline{Q_0}$\\
        $T'_1=(T_1)_{up} \cdot \overline{Q_3} \cdot \overline{Q_0} + (T_1)_{down} \cdot \overline{Q_3} \cdot \overline{Q_2} \cdot \overline{Q_1} \cdot \overline{Q_0}$\\
        $T'_2=(T_2)_{up} \cdot \overline{Q_3} \cdot \overline{Q_0} + (T_2)_{down} \cdot \overline{Q_3} \cdot \overline{Q_2} \cdot \overline{Q_1} \cdot \overline{Q_0}$\\
        $T'_3=(T_3)_{up} \cdot \overline{Q_3} \cdot \overline{Q_0} + (T_3)_{down} \cdot \overline{Q_3} \cdot \overline{Q_2} \cdot \overline{Q_1} \cdot \overline{Q_0}$\\
    \end{center}
    
    \section{Circuit Design}
    Mainly the circuit is built using logic gates and jk flip-flop
    \subsection{Components}
    \begin{itemize}
      \item \textbf{Arduino Uno} – 1 \\
      Provides 5V power and possibly input control (button pulse or clock).
    
      \item \textbf{Breadboards} – 3 \\
      Two medium-sized for ICs and one large for extra wiring space.
    
      \item \textbf{74HC73 (JK Flip-Flop)} – 6 \\
      Dual JK flip-flops with Clear; used for BCD counting and acts as T flip flop when J and K are shorted(4 bits per digit).
    
      \item \textbf{74HC00 (Quad 2-input NAND gate)} – 3 \\
      Used for implementing logic conditions.
    
      \item \textbf{74HC08 (Quad 2-input AND gate)} – 3 \\
      Used to enable logic outputs as per Karnaugh expressions.
    
      \item \textbf{74HC32 (Quad 2-input OR gate)} – 3 \\
      Used to combine logic signals.
    
      \item \textbf{74HC04 (Hex Inverter / NOT gate)} – 2 \\
      Used for inverting control and data signals.
    
      \item \textbf{CD4511 (BCD to 7-segment decoder)} – 2 \\
      Drives 7-segment displays using 4-bit BCD input.
    
      \item \textbf{7-Segment Displays} – 2 \\
      Likely common cathode; used to visually represent counter value.
    
      \item \textbf{Push Buttons} – 3 \\
      Used for manual Up, Down, and Reset controls.
    
      \item \textbf{Resistors} – approximately 10–15 \\
      Used for pull-down configurations and current limiting on LEDs.
    
      \item \textbf{Jumper Wires} – Multiple \\
      For all the required interconnections across components and breadboards.
    \end{itemize}
    \subsection{Circuit Schematic}
    \begin{figure}[h!]
        \centering
        \includegraphics[width=\textwidth]{counter_circuit.jpeg}
        \caption{Breadboard implementation of a BCD up/down counter using T flip-flops and logic gates.}
        \label{fig:bcd_counter}
    \end{figure}
    \subsection{Wiring}
    \includepdf[pages=-]{bcd counter.pdf}
    \subsection{Simulation}
    \section{Conclusion}
    The above circuit performs number counting from 0 to 99.When up button is pressed the number increases by one and down button is pressed number decreases by one
    \end{document}
